
%%%%%% 4: DESCRIZIONE RISULTATI
\section{Risultati}
Presentiamo in questa sezione i risultati per il test error e per la frazione di cifre indovinate dagli algoritmi calcolati.

\begin{figure}[H]
\centering
\includegraphics[width=\textwidth]{testerr_single}
\caption{Andamento del Test Error dei classificatori binari a singola cifra al variare di $T$.}

\includegraphics[width=\textwidth]{guessed_single}
\caption{Andamento della frazione di risposte corrette dei classificatori binari a singola cifra al variare di $T$.}
\label{fig:single}
\end{figure}


I grafici per i classificatori binari evidenziano, come atteso, un andamento pressoch\'e monotono decrescente del test error al crescere di $T$ e un corrispettivo andamento monotono crescente della frazione di cifre riconosciute. \`E interessante notare che ogni cifra ha un andamento leggermente diverso, e tende a stabilizzarsi su un diverso valore ``finale'' per $T$ elevati. Questi valori differiscono circa di un fattore 4 tra gli estremi, con la cifra 1 avente un test error vicino a 0.01 e la cifra 5 vicino a 0.04. L'ordine delle cifre, dalla pi\`u facilmente riconosciuta alla meno, \`e:

\[(1, 0, 6, 7, 4, 2, 3, 8, 9, 5)\]

\begin{figure}[h]
\centering
\includegraphics[width=\textwidth]{min_testerr}
\caption{Test Error a \(T = 250\) dei classificatori binari a singola cifra.}
\label{fig:min_testerr}
\end{figure}

%% 1 v A

\begin{figure}[h]
\centering
\includegraphics[width=\textwidth]{testerr_1vA}
\caption{Andamento del Test Error del classificatore combinato One-vs-All al variare di $T$.}

\includegraphics[width=\textwidth]{guessed_1vA}
\caption{Andamento della frazione di risposte corrette del classificatore combinato One-vs-All al variare di $T$.}
\label{fig:1vA}
\end{figure}

Per quanto riguarda il classificatore combinato, l'andamento del test error e della frazione di cifre indovinate \`e simile a quella dei classificatori binari, e il classificatore combinato ottiene circa il 90\% di successo con \(T = 250\). 