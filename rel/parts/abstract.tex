\begin{abstract}
\noindent
Nel seguente progetto ci si propone di applicare l'algoritmo Adaboost al problema del riconoscimento automatico delle cifre scritte a mano. 
Si utilizzano come {\it weak learner} dei {\it decision stump} della forma
\begin{center}
	\( h_\tau(x) = sgn(x - \tau) \)
\end{center}
Vengono quindi prodotti 10 distinti {\it strong classifier}, uno per cifra, in grado di distinguere se una data immagine \`e la cifra che sono stati allenati a riconoscere o no. Infine, usando una tecnica{\it one vs all} sui classificatori ottenuti, si produce un classificatore non binario in grado di determinare la cifra esatta data l'immagine.
Il dataset utilizzato \`e il MNIST\footnote{http://yann.lecun.com/exdb/mnist/}.
\end{abstract}