%%%%% 1: DESCRIZIONE DATI

\section{Descrizione dei dati} \label{sec:data}
I dati MNIST utilizzati nel corso del progetto sono suddivisi in due tipologie: immagini contenenti le cifre scritte a mano ed etichette che descrivono la cifra contenuta in ogni immagine.

\subsection{Immagini}
Una singola immagine consiste di un {\it array} di pixel \(\{x_0, \dotsc, x_M\}\), ognuno rappresentato da un singolo byte senza segno che contiene il "colore" del pixel in scala di grigi (ogni pixel pertanto spazia dal valore 0 al valore 255). Tutte le immagini hanno la stessa altezza e larghezza: 28 x 28 pixel.

\subsection{Etichette}
Ogni etichetta \`e rappresentata da un singolo byte senza segno che rappresenta una cifra decimale. Pertanto un'etichetta pu\`o contenere solamente valori compresi tra 0 e 9 inclusi.

\subsection{File di dati}
 Oltre alla suddivisione di tipologia sopra descritta, i dati sono ulteriormente suddivisi in dati di training ({\it training set}) e dati di test ({\it test set}). 
 Il database MNIST permette quindi di scaricare quattro diversi file:
\begin{enumerate}
	\item {\tt train-images-idx3-ubyte}: contiene le immagini di training
	\item {\tt train-labels-idx1-ubyte}: contiene le etichette associate alle immagini di training
	\item {\tt test-images-idx3-ubyte}: contiene le immagini di test
	\item {\tt test-labels-idx1-ubyte}: contiene le etichette associate alle immagini di test
\end{enumerate}
Ognuno di questi \`e intestato con un {\it header} che ne descrive il contenuto.